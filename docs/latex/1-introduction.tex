%%%%%%%%%%%%%%%%%%%%%%%%%%%%%%%%%%%%%%%%%%%%%%%%%%%%%%%%%%%%%%%%%%%%%%%%%%%%%%%
%%                                                                           %%
%%                    1 - I N T R O D U C T I O N . T E X                    %%
%%                                                                           %%
%%                                 M A L E F                                 %%
%%                                                                           %%
%%                                 L a T e X                                 %%
%%                                                                           %%
%%---------------------------------------------------------------------------%%
%%     Copyright (c) 2021 José Antonio Verde Jiménez All Rights Reserved     %%
%%---------------------------------------------------------------------------%%
%% Permission  is granted to copy,  distribute  and/or  modify this document %%
%% under the terms of the GNU Free Documentation License, Version 1.3 or any %%
%% later  version  published  by  the  Free  Software  Foundation;  with  no %%
%% Invariant Sections, no Front-Cover Texts, and no Back-Cover Texts.        %%
%% A copy  of the  license is  included in  the section  entitled  "GNU Free %%
%% Documentation License" or the files "fdl.tex" or "LICENSE.fdl-v1.3".      %%
%%                                                                           %%
%%%%%%%%%%%%%%%%%%%%%%%%%%%%%%%%%%%%%%%%%%%%%%%%%%%%%%%%%%%%%%%%%%%%%%%%%%%%%%%

\section {Introduction}
   \subsection {What is Malef?}
   \paragraph{}
      \textbf {Malef} is a 'TUI' library written in Ada with bindings for other
   languages such as C or Python. TUI stands for \textit{Text User Interface}.
   TUI programs have the advantage of being way simpler to code and have fewer
   dependencies than any other GUI application. That's the purpose of Malef,
   being able to run anywhere suiting the needs of the program to the
   capabilities of any terminal available. Hiding all the complicated stuff and
   allowing you to write any kind of application in a similar fashion to
   writing a GUI application, but with the simplicity of a TUI.

   \paragraph{}
      \textbf{Malef} is terminal and system independent. For this date it has
   been tested in both Linux and Windows, but theoreticaly it should be able to
   work in Unix too. Also, it supports both ANSI terminals and the Windows'
   CMD. In future versions more and more subsystems will be added.

   \paragraph{\textit{Note:}} \textit{
      When I talk about systems I refer to \textbf{WHERE} it is running, e.g:
   Linux, Windows, Unix or a Web Browser (in future versions). And when I talk
   about subsystems I mean \textbf{WHICH} program is used to run it, e.g:
   an ANSI-compliant terminal, a CMD, a TTY...}

   \subsection{Tell me how to get it}
   \paragraph{}
      First go to the GitHub repository in
   \url{https://github.com/joseaverde/Malef} and download it. You can either
   clone it with the \textit {git clone} command or download it from the
   releases section on the GitHub page. There are some precompiled packages
   with instructions for different platforms, but if you want to build it
   yourself continue reading.

   \paragraph{}
      Once you've downloaded it and extracted it, you have to compile it.
   Read the README.md file for more information. But basically what you have to
   do is compile the components: there you will find several files with
   \textit{.gpr} extension. To compile them just read the \textit{shared.gpr}
   file to know which variables can be set. And finally compile the components
   you want:
   \begin{itemize}
      \item \textbf{malef.gpr} This is the basic one, it's the Ada
         implementation.
      \item \textbf{c\_malef.gpr} This is the C binding, it has the Ada
         implementation as a dependency, so both will be compiled alongside.
      \item \textbf{py\_malef.gpr} This is the Python3 binding, don't forget
         to specify the Python version you're using.
      \item \textbf{malef\_db.gpr} This compiles the Malef Data Bases so it can
         recognise certain terminals and use their configuration files.
      \item \textbf{malef\_subsystems.gpr} This is used to build the different
         subsystems. For instance, you can build the ANSI subsystem and CMD
         subsystems for Windows and Malef will try to guess which one to use.
      \item \textbf{make\_all.gpr} If you want to build all, just use this
         project file.
   \end{itemize}
   \paragraph{}
      To build project files you need \textit{gprbuild}, you can get it from
   your trusty package manager or from the GNAT Community version. You can
   specify the project file with the \textit{-P} flag, and the variables with
   the \textit{-XVARIABLE\_NAME=value} flag.

   \paragraph{}
      To \textbf{install} it, just use the command
   \textit{gprinstall -P\{project file\} $--$prefix=\{install location\}}.


%%%=======================%%%%%%%%%%%%%%%%%%%%%%%%%=========================%%%
%%=======================%% E N D   O F   F I L E %%=========================%%
%%%=======================%%%%%%%%%%%%%%%%%%%%%%%%%=========================%%%
